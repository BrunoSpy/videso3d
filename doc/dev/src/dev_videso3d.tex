\documentclass[a4paper, titlepage]{article}

\usepackage[utf8x]{inputenc}
\usepackage{amsfonts}
\usepackage{amsmath}
\usepackage[frenchb]{babel}
\usepackage[T1]{fontenc}
\usepackage[dvips,pdftex]{graphicx}
\usepackage{geometry}
\geometry{a4paper,twoside,left=3cm,right=3cm,marginparwidth=1.2cm,marginparsep=3mm,top=3cm,bottom=3cm}

\title{
\vspace{1cm} 
\rule{\textwidth}{1mm}
\vspace{1mm}\\
\begin{tabular}{p{0cm} c}
   & {\Huge {\bf Videso3D}} \\
   & \\
   & {\huge Visualisation 3D} \\
   & {\huge des Données d'Exploitation} \\
   & {\huge des Systèmes Opérationnels}
 \end{tabular}\\
\rule{\textwidth}{1mm}
\vspace{2cm}\\
\begin{tabular}{p{0cm} c}
	&{\huge Manuel du développpeur}\\
   &{\large Version 0.1}
 \end{tabular}
\vspace{2cm}\\
 \begin{tabular}{p{0cm} c}
    & {\large CRNA Nord} \\
    & {\large  \today}
 \end{tabular}\\
\vspace{1cm}
}



\author{
Bruno \textsc{Spyckerelle}\\
Jonathan \textsc{Colson}}

\date{}

\begin{document}

\maketitle
\tableofcontents
\newpage

\section{A propos de ViDESO}
Videso3D est développé avec les technologies suivantes :
\begin{itemize}
 \item Java 6 ;
 \item Nasa WorldWind.
\end{itemize}
\par


Ce manuel décrit comment configurer un environnement de développement Eclipse
et comment remonter les modifications aux auteurs de Videso3D.

\section{Installer Qt Jambi}

\section{Configurer Eclipse}
\subsection{Subversive}
\subsection{Texlipse}

\section{Télécharger les sources de Videso3D}

\section{Suivi des tâches : Mantis}

\end{document}

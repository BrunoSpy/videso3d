\documentclass[a4paper, titlepage]{article}

\usepackage[utf8x]{inputenc}
\usepackage{amsfonts}
\usepackage{amsmath}
\usepackage[frenchb]{babel}
\usepackage[T1]{fontenc}
\usepackage[dvips,pdftex]{graphicx}
\usepackage{geometry}
\geometry{a4paper,twoside,left=3cm,right=3cm,marginparwidth=1.2cm,marginparsep=3mm,top=3cm,bottom=3cm}

\title{
\vspace{1cm} 
\rule{\textwidth}{1mm}
\vspace{1mm}\\
\begin{tabular}{p{0cm} c}
   & {\Huge {\bf Videso3D}} \\
   & \\
   & {\huge Visualisation 3D} \\
   & {\huge des Données d'Exploitation} \\
   & {\huge des Systèmes Opérationnels}
 \end{tabular}\\
\rule{\textwidth}{1mm}
\vspace{2cm}\\
\begin{tabular}{p{0cm} c}
	&{\huge Manuel du développpeur}\\
   &{\large Version 0.2}
 \end{tabular}
\vspace{2cm}\\
 \begin{tabular}{p{0cm} c}
    & {\large CRNA Nord} \\
    & {\large  \today}
 \end{tabular}\\
\vspace{1cm}
}



\author{
Bruno \textsc{Spyckerelle}\\
Jonathan \textsc{Colson}}

\date{}

\begin{document}

\maketitle
\tableofcontents
\newpage

\section{A propos de ViDESO}
Videso3D est développé avec les technologies suivantes :
\begin{itemize}
 \item Java 6 ;
 \item Nasa WorldWind pour la représentation 3D ;
 \item SQlite pour la base de données.
\end{itemize}
\par


Ce manuel décrit comment configurer un environnement de développement Eclipse
et comment remonter les modifications aux auteurs de Videso3D.


\section{Configurer Eclipse}
\subsection{Installer Eclipse}
Eclipse est disponible sur http://www.eclipse.org/downloads

Il faut télécharger la version \emph{IDE for Java Developers}.

Pour installer Eclipse, il suffit de décompresser l'archive à l'endroit
souhaité.
\subsection{Subversive}
Une fois Eclipse téléchargé et installé, il faut configurer le module
Subversive, qui permettra de récupérer les sources de Videso3D et d'envoyer ses
modifications.

\vspace{\baselineskip}

Pour installer Subversive :
\begin{itemize}
  \item lancer Eclipse et aller dans Help->Install New Softwares\ldots
  \item Choisir \emph{Work with: --All Available Sites--}
  \item Cocher \emph{Subversive SVN Team Provider} dans Collaboration
  \item Cliquer sur Next, accepter les licenses et finir.
\end{itemize} 

\vspace{\baselineskip}

Il faut ensuite ajouter un connecteur :
\begin{itemize}
  \item Il faut d'abord ajouter le site suivant dans \emph{Install New
  Softwares} http://community.polarion.com/projects/subversive/download/eclipse/2.0/galileo-site/
  \item Dans \emph{Subversive SVN Connectors}, choisir \emph{SVNKit 1.3.0}
  \item Cliquer sur Next, accepter les licenses et finir.
\end{itemize}

\vspace{\baselineskip}
Eclipse est prêt à créer un nouveau projet à partir des sources de Videso3D.

\subsection{Texlipse}
Afin de pouvoir modifier la documentation avec Eclipse, il faut aussi ajouter
le module Texlipse.


\section{Récupérer les sources de Videso3D avec Eclipse}
Cette section décrit comment créer un nouveau projet dans Eclipse à partir des
sources de Videso3D.

\subsection{Créer un nouveau projet Videso3D}

Créer un nouveau projet :
\begin{itemize}
  \item Aller dans File->New->Project\ldots
  \item Choisir SVN->Project from SVN
\end{itemize}

\vspace{\baselineskip}
Configurer l'accès svn:
\begin{itemize}
  \item Dans URL, mettre \emph{https://videso3d.googlecode.com/svn}
  \item Cliquer sur Browse pour vérifier que tout marche. Le cas échéant, une
  arborescence du type branches, tags, trunk devrait apparaitre.
  \item (Facultatif) Remplir les données d'authentification pour avoir un
  accés en écriture sur le projet.
  \item A l'étape \emph{Select Resource}, parcourir l'arborescence et choisir
  le répertoir \emph{trunk}, laisser la révision sur HEAD.
  \item A l'étape \emph{Check Out As}, choisir \emph{Check out as a project
  configured using the New Project Wizard}
  \item Cliquer sur Finish et choisir Java->Java Project
  \item Mettre Videso3D comme \emph{Project name} et cliquer sur Finish.
\end{itemize}

\subsection{Configurer le projet}
Une fois le projet récupéré, il faut le configurer afin qu'il compile
correctement :
\begin{itemize}
  \item Aller dans Project->Properties
  \item Puis aller dans Java Build Path->Libraries
  \item Cliquer sur Add Jars
  \item Ajouter depuis lib les jars suivants : gluegen-rt.jar, jogl.jar,
  lablib-checkboxtree-3.1.jar, sqlite-jdbc-3.6.17.3.jar, worldwind.jar et
  swingx-1.0.jar
  \item Dans jogl.jar, configurer \emph{Native library location} à
  \emph{Videso3D/lib} (Workspace, puis choisir lib)
  \item Enfin, dans Resource, mettre \emph{Text file encoding} sur Other->UTF-8
\end{itemize}

\subsection{Lancer Videso3D}
Pour lancer Videso3D, ouvrir le fichier fr.crnan.videso3d.Videso3D.java et
lancer en tant que Java Application (clic sur la flèche verte).

\subsection{Organisation des sources}
Les sources de Videso3D sont organisées selon l'arborescence suivante :
\begin{itemize}
  \item $\left[Supprim\acute{e}\right]$ \emph{datas} : Contient un jeu de
  données à des fins de test.
 		\begin{itemize}
           \item \emph{091202\_v7}
           \item \emph{CA120209}
           \item \emph{Carto\_du\_09\_04\_2009}
        \end{itemize}
  \item \emph{doc} : Contient la documentation relative à Videso3D
  		\begin{itemize}
           \item \emph{api} : la javadoc
           \item \emph{dev} : le manuel du développeur
           \item $\left[Supprim\acute{e}\right]$ \emph{import} : les différents
           manuels concernant l'import de fichiers
           \item \emph{mut} : le manuel utilisateur
        \end{itemize} 
  \item \emph{lib} : les bibliothèques utilisées par Videso3D (notamment
  WorldWind)
  \item \emph{src} : les sources de Videso3D
\end{itemize}
Les dossiers \emph{datas} et \emph{import} ont été supprimé lors du passage à
Google Code, car leur statut n'est pas très clair.
\section{Google Code}
Le projet est hébergé sous Google Code à l'adresse suivante :
\emph{http://videso3d.googlecode.com}.

\end{document}
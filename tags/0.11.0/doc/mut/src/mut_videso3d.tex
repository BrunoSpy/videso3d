\documentclass[a4paper, titlepage]{article}

\usepackage[utf8x]{inputenc}
\usepackage{amsfonts}
\usepackage{amsmath}
\usepackage[frenchb]{babel}
\usepackage[T1]{fontenc}
\usepackage[dvips,pdftex]{graphicx}
\usepackage{geometry}
\geometry{a4paper,twoside,left=3cm,right=3cm,marginparwidth=1.2cm,marginparsep=3mm,top=3cm,bottom=3cm}

\title{
\vspace{1cm} 
\rule{\textwidth}{1mm}
\vspace{1mm}\\
\begin{tabular}{p{0cm} c}
   & {\Huge {\bf Videso3D}} \\
   & \\
   & {\huge Visualisation 3D} \\
   & {\huge des Données d'Exploitation} \\
   & {\huge des Systèmes Opérationnels}
 \end{tabular}\\
\rule{\textwidth}{1mm}
\vspace{2cm}\\
\begin{tabular}{p{0cm} c}
	&{\huge Manuel Utilisateur}\\
   &{\large Version 0.1}
 \end{tabular}
\vspace{2cm}\\
 \begin{tabular}{p{0cm} c}
    & {\large CRNA Nord} \\
    & {\large  \today}
 \end{tabular}\\
\vspace{1cm}
}



\author{
Bruno \textsc{Spyckerelle}\\
Jonathan \textsc{Colson}}

\date{}

\begin{document}

\maketitle
\tableofcontents
\newpage

\section{Présentation}
\subsection{Qu'est-ce que Videso3D ?}
Videso3D est un logiciel de visualisation 3D des données d'exploitation des
systèmes opérationnels.

A terme, Videso3D sera capable d'afficher n'importe quelle DESO, de faire des
recherches dans ces données et de comparer différentes versions.

\subsection{Données lues par Videso3D}
\subsubsection{Données Stip}
\paragraph{Partitions importées \newline}
Les partitions suivantes sont importées par Videso3D :
\begin{itemize}
  \item CENTRE
  \item SECT
  \item balise
  \item POINSECT
  \item ROUTSECT
  \item ROUTE
\end{itemize}
\paragraph{Cas particulier des données concernant les frontières \newline}
Les données Stip concernant les frontières des pays ne sont pas livrées par le
CESNAC lors d'une livraison d'une bande CA. Pour cette raison, ces données sont
fournies avec Videso3D mais traitées de la même façon que d'autres DESO.
\newline
Ces données apparaissent dans le gestionnaire de données en tant que données
PAYS.
 
\subsubsection{Données STR Exsa}
Videso3D affiche les données STR suivantes :
\begin{itemize}
  \item les mosaïques radar;
  \item les VVF ;
  \item les zones d'occultation ;
  \item les zones de filtrage capacitif ;
  \item les zones de filtrage dynamique.
\end{itemize}
\subsubsection{Données Stpv}
Pour l'instant, Videso3D n'importe aucune donnée STPV.
\subsubsection{Données Edimap}
Videso3D sait afficher les cartes Edimap au format Nectar.
\section{Installation}
Videso3D nécessite une machine virtuelle Java 6.

Si Java 6 est déjà installé, il suffit de décompresser l'archive
\emph{videso-x.x.zip} dans le répertoire voulu.

\section{Configuration}


\section{Utilisation}
\subsection{Ajouter des données}
Pour ajouter des données, il faut se rendre dans Fichier>Ajouter des données.
\subsection{Sélectionner les données à afficher}

\subsection{Faire une recherche}
\subsubsection{Rechercher un item affiché sur la carte}
\section{Questions fréquentes}

\end{document}
